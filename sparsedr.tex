%%%% ijcai26.tex

\typeout{IJCAI--ECAI 26 Instructions for Authors}

% These are the instructions for authors for IJCAI--ECAI 26.

\documentclass{article}
\pdfpagewidth=8.5in
\pdfpageheight=11in

% The file ijcai26.sty is a copy from ijcai22.sty
% The file ijcai22.sty is NOT the same as previous years'
\usepackage{ijcai26}

% Use the postscript times font!
\usepackage{times}
\usepackage{soul}
\usepackage{url}
\usepackage[hidelinks]{hyperref}
\usepackage[utf8]{inputenc}
\usepackage[small]{caption}
\usepackage{graphicx}
\usepackage{amsmath}
\usepackage{amsthm}
\usepackage{booktabs}
\usepackage{algorithm}
\usepackage{algorithmic}
\usepackage[switch]{lineno}

% Comment out this line in the camera-ready submission
\linenumbers

\urlstyle{same}

% the following package is optional:
%\usepackage{latexsym}

% See https://www.overleaf.com/learn/latex/theorems_and_proofs
% for a nice explanation of how to define new theorems, but keep
% in mind that the amsthm package is already included in this
% template and that you must *not* alter the styling.
\newtheorem{example}{Example}
\newtheorem{theorem}{Theorem}

% Following comment is from ijcai97-submit.tex:
% The preparation of these files was supported by Schlumberger Palo Alto
% Research, AT\&T Bell Laboratories, and Morgan Kaufmann Publishers.
% Shirley Jowell, of Morgan Kaufmann Publishers, and Peter F.
% Patel-Schneider, of AT\&T Bell Laboratories collaborated on their
% preparation.

% These instructions can be modified and used in other conferences as long
% as credit to the authors and supporting agencies is retained, this notice
% is not changed, and further modification or reuse is not restricted.
% Neither Shirley Jowell nor Peter F. Patel-Schneider can be listed as
% contacts for providing assistance without their prior permission.

% To use for other conferences, change references to files and the
% conference appropriate and use other authors, contacts, publishers, and
% organizations.
% Also change the deadline and address for returning papers and the length and
% page charge instructions.
% Put where the files are available in the appropriate places.


% PDF Info Is REQUIRED.

% Please leave this \pdfinfo block untouched both for the submission and
% Camera Ready Copy. Do not include Title and Author information in the pdfinfo section
\pdfinfo{
/TemplateVersion (IJCAI.2026.0)
}

\title{Differentiable Rendering of Sparse SDFs}

% Single author syntax
% \author{
%     Author Name
%     \affiliations
%     Affiliation
%     \emails
%     email@example.com
% }

% Multiple author syntax (remove the single-author syntax above and the \iffalse ... \fi here)
% \iffalse
\author{
First Author$^1$
\and
Second Author$^2$\and
Third Author$^{2,3}$\And
Fourth Author$^4$\\
\affiliations
$^1$First Affiliation\\
$^2$Second Affiliation\\
$^3$Third Affiliation\\
$^4$Fourth Affiliation\\
\emails % usually included
\{first, second\}@example.com,
third@other.example.com,
fourth@example.com
}
% \fi

\begin{document}

\maketitle

\begin{abstract}
    % 3D реконструкция получила заметное развитие в последнее десятилетие с появлением
    % дифференцируемого рендеринга -- мощного инструмента, породившего большое разнообразие
    % методов. Одна группа методов направлена на восстановление объектов, представленных
    % SDF сетками, в рамках физически-обоснованного рендеринга. Большим недостатком
    % такого подхода является быстрый рост потребляемой памяти при восстановлении моделей
    % с fine details. В этой работе мы предлагаем адаптацию метода дифференцируемого
    % рендеринга SDF под разреженное представление, с разреженными версиями редистансинга
    % и регуляризации. Благодаря этому, мы получаем сокращение используемой памяти и
    % более точное восстановление геометрии объектов.

    3D reconstruction has achieved notable development over the past decade with the emergence of differentiable rendering -- a powerful tool that has spawned a wide variety of methods. One group of methods is aimed at reconstructing objects represented by SDF grids within the framework of physically-based rendering. A major drawback of such an approach is the rapid growth of memory consumption when reconstructing models with fine details. In this work, we propose an adaptation of the differentiable rendering method for SDF to a sparse representation, with sparse versions of redistancing and regularization. Through this, we achieve a reduction in memory usage and more accurate reconstruction of object geometry.
\end{abstract}

\section{Introduction}
    3D object reconstruction as (something else and then) Inverse rendering.
    % Задача трёхмерной реконструкции -- получение информации об объекте реального
    % мира, в частности его форме. (TODO: про 3д реконструкцию + подводка к обратному рендерингу)

    % В рамках компьютерного зрения и графики ещё в конце XX века [ссылка] была сформулирована
    % задача обратного рендеринга. Если традиционный, прямой рендеринг позволяет получить
    % 2D изображение из 3D сцены, то обратный рендеринг подразумевает восстановление
    % параметров исходной сцены по её изображениям. Таким образом, задача обратного рендеринга
    % включает в себя трёхмерную реконструкцию.

    Inverse rendering to Differentiable rendering.
    % В последнее десятилетие для решения задачи обратного рендеринга активно
    % используется дифференцируемый рендеринг -- метод, восстанавливающий объекты
    % по изображениям с использованием градиентных методов оптимизации. Существует
    % большое число разнообразных методов дифференцируемого рендеринга. Между собой
    % они различаются, в первую очередь, тем, какое представление используется в процессе
    % оптимизации: нейронные SDF [NeuralAngelo+], NeRF [ссылки], 3DGS [ссылки] (TODO: ещё что-нибудь).

    DR methods for physically-based rendering. Tell a bit about meshes and Nicolet article, then about two nonNN methods (2022,2024).
    % В 2018 году появился метод edge sampling [ссылка], впервые позволивший получить
    % аналитические производные по параметрам геометрии, в частности меша, в физически-обоснованном
    % рендеринге. В этих условиях цвет пикселя представляет собой интеграл. При дифференцировании
    % по параметрам, задающим форму объекта, подынтегральное выражение терпит разрывы,
    % положение которых зависит от этих параметров. Для корректного расчёта градиентов
    % требуется оценить интеграл по этим разрывам, названный граничным интегралом. В edge
    % sampling предлагалось явно семплировать граничный интеграл, что требовало нахождения
    % всех граней, образующих силуэт объекта. В 2019
    % и 2020 годах появились два связанных между собой метода, которые избавлялись от
    % необходимости явного семплирования границ объекта при оценке производных. Первый
    % подход, репараметризация [ссылка], предлагал замену переменных, которая локально
    % учитывала разрывы, что позволяло уйти от оценки граничного интеграла вовсе. Метод
    % 2020 года, warped area sampling (WAS) [ссылка], строго показал, что граничный
    % интеграл в формуле производной возникает согласно транспортной теореме Рейнольдса,
    % и предложил продлить область определения граничного интеграла на пространство с
    % помощью теоремы Остроградского-Гаусса. Для этого перехода строится специальное 
    % векторное поле.

    % Три подхода к дифференцируемому рендерингу, описанные выше, использовали меши в практической 
    % реализации. Однако, меши, с точки зрения задачи реконструкции поверхности, имеют ряд фундаментальных 
    % проблем, которые делают их использование нежелательным. Их недостатки хорошо описаны и 
    % проиллюстрированы в работе 2021 года [ссылка]. Фундаментальной проблемой мешей является 
    % невозможность менять топологию объекта (род поверхности), также во время оптимизации могут возникать 
    % необратимые самопересечения и неоптимальные распределения примитивов, из-за чего в местах, требующих 
    % высокой детализации, нет достаточного числа примитивов для их представления. Модификации, 
    % представленные в работе, частично решают две последние проблемы, но
    % ограничение на топологию мешает свободному применению мешей для оптимизации.

    % Следом появилось несколько методов, также применимых в физически-обоснованном
    % рендеринге, которые использовали SDF grids в качестве оптимизируемого представления.
    % В 2022 году к SDF впервые применили репараметризацию [ссылка]. В этой работе было
    % предложено использовать алгоритм редистансинга. Редистансинг – задача пересчёта
    % значений сетки, чтобы её вершины содержали корректные расстояния до поверхности. В
    % дифференцируемом рендеринге пересчёт нужен, поскольку после шага оптимизации и 
    % обновления расстояний в узлах сетки она, строго говоря, уже не является SDF. Наконец,
    % метод 2024 года [ссылка] использует свойства функций расстояния и
    % оценивает граничный интеграл интегралом по релаксированной границе – тонкой линии
    % вокруг силуэта объекта.

    This method uses the idea of a relaxed boundary and the redistancing and regularization
    steps, necessary for SDF-based DR methods.
    % TODO: явно сказать, что новое: разреженное представление с другим алгоритмом обхода,
    % а также разреженные версии редистансинга и регуляризации.

\section{System Architecture} % Что.
    Describe architecture -- components (their algorithms).
    % TODO: Структура в целом: расчёт производных (с того, что процесс итеративный, до
    % использования метода Ньютона для поиска пересечения и точек релаксированной границы).
    % Редистансинг, регуляризация, разрежение+апсемплинг.

    % (может быть) ПРИВЕСТИ ДВА АЛГОРИТМА: ОДИН ДЛЯ ЦЕЛОГО ЭТАПА, ОТ АПСЕМПЛИНГА ДО АПСЕМПЛИНГА,
    % ВТОРОЙ: ОДНА ИТЕРАЦИЯ ДИФФ РЕНДЕРИНГА
    Tell about SDFs, grids, then (2022 article about SVS-SBS-etc.).
    This method uses SBS and Newton method for intersection and determining relaxed
    boundary points.
    % TODO: рассказать, что хранится в SBS, а что отбрасывается, сказать, что добавление бриков в процессе пока не добавлено.

    % Рассказать про редистансинг, регуляризацию, разрежение и апсемплинг.

\section{Implementation} % Как.
    Describe technical implementation. Comparisons also go here.

    \subsection{Sparse SDF Representaiton}
    % TODO: 

    \subsection{Upsampling and sparsification} % Find a better word

    \subsection{Redistancing}
    % TODO: описать ВСЁ - этапы, как они устроены

    \subsection{Regularization}
    % TODO: сделать регуляризацию, дать описание

     

\subsection{Order of Sections}

Sections should be arranged in the following order \cite{bgf:Lixto}:
\begin{enumerate}
    \item Main content sections (numbered)
    \item Appendices (optional, numbered using capital letters)
    \item Ethical statement (optional, unnumbered)
    \item Acknowledgements (optional, unnumbered)
    \item Contribution statement (optional, unnumbered)
    \item References (required, unnumbered)
\end{enumerate}


%% The file named.bst is a bibliography style file for BibTeX 0.99c
\bibliographystyle{named}
\bibliography{sparsedr}

\end{document}

